\documentclass[11pt,a4paper,twoside]{article}
\usepackage{tascar}
\pagenumbering{arabic}
\pagestyle{empty}
\showtutorialtrue
\begin{document}
\setcounter{tutorial}{-1}
\begin{tutorial}{\tascar{} basics}

  \begin{learnitems}
  \item How to create a scene
  \end{learnitems}

  \begin{appitems}
  \item Basis for all other tutorials
  \end{appitems}

\end{tutorial}

\begin{itemize}
\item The real-time version of \tascar{} is heavily build upon the
  jack audio connection kit (jack,
  \url{http://www.jackaudio.org}). Familiarize yourself with jack
  and its tools. We recommend to use the \verb!patchage! tool for
  visualizing the signal graph.
\item Open \textbf{user manual}
  (\url{file:///usr/share/doc/tascar/manual.pdf}). You
  can use the manual as a help throughout the workshop.

\item Open directory with \tascar{} scenes:
  \verb!/home/medi/tascar_scenes/! -- you will find many examples of
  \tascar{} scenes as well as tons of sound samples there. The
  examples from the manual can be found in
  \verb!/usr/share/tascar/examples!.

\item Open text editor (e.g., \verb!gedit!) and open the basic example
  \verb!basic_example.tsc!. You can create your own copy, e.g.
  \verb!Group1_Task1.tsc!. This can be your own scene-definition xml
  file.

\item Start the sound server \verb!jack! (e.g., via qjackctl).

\item Load your scene into \tascar{}. Compare your scene definition
  file with what you see in the \tascar{} window. Try to identify
  objects in the scene, see how they are defined, what parameters do
  they have. You can always change/add/modify things in the scene and
  see what happens. Press the ``reload'' button.

\item Visualize the audio signal flow chart with \verb!patchage!  and
  have a look at the audio ports. Try to recognize the ports
  corresponding to sources and receiver in the scene. Try to
  connect/disconnect ports.

\item You can decide about the content of the scene yourself. Use
  chapter 3-5 from the \textbf{user manual} -- you will find an
  introduction to how to create a scene there, look at other \verb!.tsc!
  files in \verb!tascar_scenes! directory. You can put any of these
  objects to your scene:
  \begin{itemize}
  \item point sources (\verb!<src_object ...><sound/></src_object>!) (5.3)
  \item diffuse sources (\verb!<diffuse ...>!) (5.4)
  \item receiver (\verb!<receiver .../>!) (5.5)
  \item reflectors (\verb!<face .../>! or \verb!<facegroup .../>!) (5.6)
  \item absorbing obstacles (\verb!<obstacle .../>!) (5.7)
  \end{itemize}

  Your objects can also move or change orientation. You can assign any
  of the available sound samples to your sources.  You can create
  different rooms and you can try out different receivers. Be creative
  and have fun!

\end{itemize}
\end{document}
