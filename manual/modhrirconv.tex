The module {\bf hrirconv} is intended for convolution of multi-channel
loudspeaker signals with head related impulse responses (HRIR), to
generate signals for binaural listening or hearing aid processing. To activate the module, add
\begin{lstlisting}[numbers=none]
<hrirconv>
  ...
</hrirconv>
\end{lstlisting}
to your session configuration.

\begin{tscattributes}
\indattr{id}          & Name used for jack                                                           \\
\indattr{fftlen}      & FFT length (need to be longer than jack fragment size)                       \\
\indattr{inchannels}  & Number of input channels                                                     \\
\indattr{outchannels} & Number of output channels                                                    \\
\indattr{autoconnect} & Auto-connect input to all receivers with matching channel count (true|false) \\
\indattr{connect}     & Input port connections (port name globbing possible)                         \\
\indattr{hrirfile}    & file name of HRIR file, channel order i1o1,i1o2,i1o3,...,i2o1,...            \\
\end{tscattributes}

The convolution matrix can be defined with the \elem{entry}
element. Each entry defines a convolution with an impulse response;
typically a convolution for each combination of input channel and
output channel is defined. The recognized attributes of \elem{entry}
are:

\begin{tscattributes}
\indattr{in}      & Input channel number (zero-based)             \\
\indattr{out}     & Output channel number (zero-based)            \\
\indattr{file}    & File name of impulse response                 \\
\indattr{channel} & File channel of impulse response (zero-based) \\
\end{tscattributes}

A typical configuration for binaural listening can look like this:
\tscexample{example_hrirconv}

A configuration file for binaural convolution together with a binaural
head model HRIR set \citep{Duda1993} can be created with the
Matlab/GNU Octave script ``tascar\_hrir\_duda.m'' (in
/usr/share/tascar/matlab). See documentation of the script for
details.

