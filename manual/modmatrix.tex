Create a jack based matrix multiplication, e.g., for Ambisonics decoding.

\begin{tscattributes}
\indattr{id} & Jack identifier\\
\indattr{decoder} & Empty (for explicit matrix), or maxre2d (for 2D-HOA max-rE decoding)\\
\end{tscattributes}

Outputs are defined as in speaker based layout files, except that they
use the element \elem{output}.
%
In addition, each speaker can contain the attribute \indattr{m}, which
contains a list of floating point values.
%
Each output channel is the sum of the product of \attr{m} with the
corresponding input channel.

Inputs are defined by the sub-elements \elem{input}.
%
Each input can have the attribute \indattr{connect}.

An Ambisonics decoder configuration can be created with the MATLAB/GNU
Octave script \verb!tacsar_generatedecmatrix.m!.

Example:
\begin{lstlisting}[numbers=none]
<matrix id="dec" decoder="maxre2d">
  <input connect="hoa:out.0" label=".0_0"/>
  <input connect="hoa:out.1" label=".1_-1"/>
  <input connect="hoa:out.2" label=".1_1"/>
  <input connect="hoa:out.3" label=".2_-2"/>
  <input connect="hoa:out.4" label=".2_2"/>
  <input connect="hoa:out.5" label=".3_-3"/>
  <input connect="hoa:out.6" label=".3_3"/>
  <input connect="hoa:out.7" label=".4_-4"/>
  <input connect="hoa:out.8" label=".4_4"/>
  <input connect="hoa:out.9" label=".5_-5"/>
  <input connect="hoa:out.10" label=".5_5"/>
  <input connect="hoa:out.11" label=".6_-6"/>
  <input connect="hoa:out.12" label=".6_6"/>
  <output az="12" connect="render.tostereo:in.0"/>
  <output az="36" connect="render.tostereo:in.1"/>
  <output az="60" connect="render.tostereo:in.2"/>
  <output az="84" connect="render.tostereo:in.3"/>
  <output az="108" connect="render.tostereo:in.4"/>
  <output az="132" connect="render.tostereo:in.5"/>
  <output az="156" connect="render.tostereo:in.6"/>
  <output az="180" connect="render.tostereo:in.7"/>
  <output az="204" connect="render.tostereo:in.8"/>
  <output az="228" connect="render.tostereo:in.9"/>
  <output az="252" connect="render.tostereo:in.10"/>
  <output az="276" connect="render.tostereo:in.11"/>
  <output az="300" connect="render.tostereo:in.12"/>
  <output az="324" connect="render.tostereo:in.13"/>
  <output az="348" connect="render.tostereo:in.14"/>
</matrix>
\end{lstlisting}

An example with explicit matrix element definitions:
\begin{lstlisting}[numbers=none]
<matrix id="mix_out">
  <input label="proc1_l" connect="adm1:out_1"/>
  <input label="proc1_r" connect="adm1:out_2"/>
  <input label="proc2_l" connect="adm2:out_1"/>
  <input label="proc2_r" connect="adm2:out_2"/>
  <output label="S_out_l" m="0.5 0 0.5 0"/>
  <output label="S_out_r" m="0 0.5 0 0.5"/>
  <output label="N_out_l" m="-0.5 0 0.5 0"/>
  <output label="N_out_r" m="0 -0.5 0 0.5"/>
</matrix>
\end{lstlisting}
