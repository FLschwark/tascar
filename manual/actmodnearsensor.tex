\begin{tscattributes}
\indattr{url} & target OSC url\\
\indattr{ttl} & time-to-live of UDP packets\\
\indattr{pattern} & pattern ob objects to detect\\
\indattr{parent} & name of parent object (= sensor position)\\
\indattr{radius} & sensor radius in meter\\
\indattr{mode} & operation mode: 0 = detect object origin, 1 = detect sound vertex\\
\indattr{path} & OSC message target path \\
\end{tscattributes}

Emit an OSC message when an object or sound vertex is near the parent object.
%
The OSC message can be composed from sub-elements of the types
\elem{f} (e.g., \attr{<f v="1.0"/>}) (float), \elem{i} 
(\attr{<i v="123"/>}) (integer) or \elem{s} (\attr{<s v="abc"/>}) (string).
%
Multiple sub-elements are possible.

Any number of sub-elements \elem{msgapp} (messages to be sent on
approaching a target) and \elem{msgdep} (messages to be sent on
departing from a target) are possible.
%
Each message has the attribute
\begin{tscattributes}
\indattr{path} & OSC message target path \\
\end{tscattributes}
and the same sub-elements \elem{f}, \elem{i} and \elem{s} as described
before.


