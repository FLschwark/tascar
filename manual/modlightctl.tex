\index{light control}\index{DMX}\index{artnetDMX}\index{Entec openDMX}

The module has these attributes:
\begin{tscattributes}
\indattr{fps} & frames per second\\
\indattr{universe} & destination universe\\
\indattr{driver} & driver (``artnetdmx'' or ``opendmxusb'')\\
\indattr{hostname} & artnetDMX destination hostname\\
\indattr{port} & artnetDMX destination port\\
\indattr{device} & openDMX USB serial device name\\
\end{tscattributes}

One or more \refelem{lightscene} elements can be defined, with these
attributes:
\begin{tscattributes}
\indattr{name}     & lightscene name                            \\
\indattr{objects}  & Tracking source object name pattern        \\
\indattr{parent}   & light center parent object name            \\
\indattr{channels} & number of channels per fixture             \\
\indattr{master}   & Master control                             \\
\indattr{method}   & Rendering method (nearest or raisedcosine) \\
\indattr{objval}   & Starting object color values.              \\
\indattr{objw}     & Starting object width values.              \\
\indattr{layout}   & fixture layout file name (optional)        \\
\end{tscattributes}

Fixtures are defined using the \indattr{fixture} element within the
\indattr{fixtures} element.
%
Syntax is the same as for speaker layout definitions, with these
additional attributes for each element:
\begin{tscattributes}
\indattr{addr}    & DMX start address                                                          \\
\indattr{dmxval}  & fixture DMX base value                                                     \\
\end{tscattributes}
For each fixture, sub-elements in the form \attr{<calib channel="0"
  in="255" out="127"/>} can be provided, to calibrate the input-output
function of the lamps. The attributes \attr{in} needs to be larger
than zero, the attributes \attr{channel} and \attr{out} need to be
larger or equal to zero.
