The module {\bf geopresets} allows to define preset positions and
orientations of objects.
%
Objects are moved to the defined preset delta-transformation following
a von-Hann ramp from the current delta-transformation to the new
delta-transformation.
%
\tscexample[linerange={3-15},firstnumber=3]{example_geopresets}

In this example, the presets ``pos'', ``posrot'' and ``rot'' can be
reached with OSC commands, e.g.,
%
\begin{verbatim}
/geopresets pos
\end{verbatim}
%
The enable state and the duration can be controlled via OSC.
%
\begin{verbatim}
/geopresets/enable 1
/geopresets/duration 3
\end{verbatim}
%

To use {\bf geopresets} in combination with the simplecontroller
(Section \ref{sec:simplecontroller}) or joystick
(Section \ref{sec:joystick}) actor plugin, configure this module to
appear before the others in the session file, and
set \attr{unlock="true"}.

\begin{tscattributes}
\indattr{duration}    & Duration of ramp in seconds (default: 2)              \\
\indattr{enable}      & Enable (true, default) or disable (false) the module. \\
\indattr{id}          & ID used as OSC prefix (default: geopresets)           \\
\indattr{startpreset} & Starting preset (or empty for no starting preset)     \\
\indattr{unlock}      & Unlock delta transformation after motion              \\
\indattr{showgui}     & Show GUI (default: false)                             \\
\indattr{width}       & Window width in pixels (default: 200)                 \\
\indattr{buttonhight} & Button height in pixels                               \\
\end{tscattributes}

Presets can be defined with one or more \elem{preset} elements, which
support these attributes:
\begin{tscattributes}
\indattr{name}        & Preset name             \\
\indattr{position}    & Position (optional).    \\
\indattr{orientation} & Orientation (optional). \\
\end{tscattributes}
