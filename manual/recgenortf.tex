ORTF stereo microphone simulation.
\index{ORTF stereo microphone}\index{stereo}
 
\begin{lstlisting}[numbers=none]
<receiver type="ortf" f6db="3000" fmin="80" distance="0.17" angle="110"/>
\end{lstlisting}

This receiver implements a classical ORTF stereo microphone
technique. The cardioid microphone pattern has a frequency dependency;
the 6~dB cutoff frequency for 90 degrees is given by the attribute
\indattr{f6db}. The attribute \indattr{fmin} defines the cutoff
frequency for 180 degree sounds. The attributes \indattr{distance} and
\indattr{angle} control the microphone geometry.

Typical values for small diaphragms are \attr{f6db="3000"} and
\attr{fmin="800"} (these are the default values since version
0.172.2); for a higher directivity, use \attr{f6db="1000"} and
\attr{fmin="60"} (default values of earlier versions).

\begin{tscattributes}
\indattr{distance} & Microphone distance in meter (0.17)\\
\indattr{angle} & Angular distance between microphone axes in degree (110)\\
\indattr{f6db}& 6~dB cutoff frequency for 90 degrees in Hz (3000)\\
\indattr{fmin}  & Cutoff frequency for 180 degree sounds in Hz (800)\\
\indattr{sincorder} & Sinc interpolation order of ITD delay line (0)\\
\indattr{c} & Speed of sound in m/s (340)\\
\indattr{decorr\_length} & Decorrelation length in seconds (0.05)\\
\indattr{decorr} & Flag to use decorrelatin of diffuse sounds (false)\\
\end{tscattributes}
