% add documentation of glab sensors...

The module {\em glabsensors} provides an interface to error reporting
in sensor drivers, and it can load drivers for various sensors.

\begin{tscattributes}
\indattr{x}, \indattr{y}, \indattr{w}, \indattr{h}       & Window position and size \\
\indattr{ontop}   & Create a window which stays on top                    \\
\end{tscattributes}

Currently, the sensors {\em trackir}, {\em eog}, {\em emergency}, {\em jackstatus} and
{\em smiley} are supported.

The {\bf trackir} sensor (optical marker tracking) supports these
attributes:
\begin{tscattributes}
\indattr{name}          & Sensor name (default: ``trackir'')                             \\
\indattr{linethreshold} & Maximal deviation from line (default: 1)                       \\
\indattr{maxdist}       & Maximal distance error of marker 2D projection (default: 0.05) \\
\indattr{margin}        & Warning margin in pixels (default: 100)                        \\
\indattr{use\_calib}    & Use camera calibration (default: true)                         \\
\indattr{flipx}         & Flip $x$ coordinate, required for some TrackIR models          \\
                        & (default: false)                                               \\
\indattr{flipy}         & Flip $y$ coordinate, required for some TrackIR models          \\
                        & (default: false)                                               \\
\indattr{f}             & Focal length of camera (default: 640)                          \\
\indattr{maxframedist}  & Maximal distance between consecutive frames for warnings       \\
                        & (default: 0.05)                                                \\
\indattr{camcalibfile}  & Name of camera calibration file                                \\ 
                        & (default: ``\$\{HOME\}/tascartrackircamcalib.txt'')            \\
\indattr{crownfile}     & Name of camera calibration file                                \\
                        & (default: ``\$\{HOME\}/tascartrackircrown.txt'')               \\
\indattr{camview}       & Draw camera view (default: true)                               \\
\end{tscattributes}
The module provides three LSL streams: {\tt trackir} contains 6
channels (translation xyz, rotation zyx) as provided by the underlying
openCV camera solving algorithm. {\tt trackirpresolve} contains also 6
channels with translation and rotation, but based on the 2-dimensional
projection. The rotation around x and y will always be zero. This
estimation might be more robust than the camera solving algorithm
based estimation in some conditions. {\tt trackirmarker} contains 30
channels, with the camera position (x,y) and pixel size of up to 10
markers. Untracked markers contain a pixel size of 0.

Camera calibration can be provided in a simple text file. White space
will be ignored, comments are allowed after the '\#' comment
character. The filed must contain six numbers: Camera position (x,y,z)
and camera Euler orientation (z,y,x). If camera calibration is
provided in an external file and locally in the XML configuration,
then the data from the external file is used.

The {\bf eog} sensor is a bluetooth serial stream based device, with
these attributes:
\begin{tscattributes}
\indattr{device}   & Serial device (default: ``/dev/rfcomm1'') \\
\indattr{baudrate} & Baud rate (default: 38400)                \\
\indattr{charsize} & Character size (default: 8)               \\
\indattr{offset}   & Data offset (default: 512)                \\
\indattr{scale}    & Data scale (default: 0.0032227)           \\
\indattr{range}    & Data range (default: ``0 1023'')          \\
\indattr{unit}     & Data unit (default: mV)                   \\
\end{tscattributes}
The LSL output stream contains one channel.

The {\bf midicc} sensor receives MIDI CC messages:
\begin{tscattributes}
\indattr{connect}     & Connect input to this MIDI port                                                                                                                           \\
\indattr{range}       & Value range mapping, input values of 0 are mapped to the first element, input values of 127 are mapped to the second, with linear interpolation (``0 1'') \\
\indattr{controllers} & Channel/Parameter pairs of controllers to receive                                                                                                         \\
\indattr{data}        & Start values sent to device upon initialization                                                                                                           \\
\end{tscattributes}

The {\bf serial} sensor reads data from the serial device, with these
attributes:
\begin{tscattributes}
\indattr{device}   & Serial device (default: ``/dev/ttyS0'') \\
\indattr{baudrate} & Baud rate (default: 38400)              \\
\indattr{charsize} & Character size (default: 8)             \\
\indattr{offset}   & Data offset (default: 0)                \\
\indattr{scale}    & Data scale (default: 1)                 \\
\indattr{channels} & Number of channels (default: 1)         \\
\end{tscattributes}
The LSL output stream contains one channel.

The {\bf emergency} sensor reacts on continuous OSC messages on path {\tt /noemergency}, and
executes a command when no OSC message arrives within a given timeout.
\begin{tscattributes}
\indattr{timeout}     & Timeout in seconds after which the command will be executed.            \\
\indattr{startlock}   & Time to wait before activation, in seconds.                             \\
\indattr{on\_timeout} & System command to be executed upon a timeout event.                     \\
\indattr{on\_alive}   & System command to be executed if the incoming messages are active again \\
\end{tscattributes}

\label{sec:espheadtracker}For the custom-made ESP-based combined head tracking and EOG
amplifier of the Gesture Lab, the module {\bf espheadtracker} was
developed. This module requires a session port number of 9800 to work,
since the port number is hard-coded into the firmware of the sensor.
\begin{tscattributes}
\indattr{timeout} & Time out for re-connection/re-initialization of LSL stream in seconds\\
\end{tscattributes}

The {\bf jackstatus} sensor analyses the JACK backend performance
(xruns and CPU load). Warnings are issued if xruns occur or if the CPU
load is above the given threshold. Critical errors are issued when the
average xrun frequency is above the given threshold, or if the CPU
load is above the threshold.
\begin{tscattributes}
\indattr{warnload}     & CPU load threshold for warnings, in percent (default: 70)        \\
\indattr{criticalload} & CPU load threshold for critical errors, in percent (default: 95) \\
\indattr{maxxrunfreq}  & Critical average xrun frequency threshold in Hz (default: 0.1)    \\
\end{tscattributes}


The {\bf smiley} sensor is used for testing and learning only. It has no configurable attributes.

